\documentclass{article}

\usepackage{amsthm}
\usepackage{amsmath}
\usepackage{amssymb}
\usepackage{geometry}
\usepackage{hyperref}

\usepackage{color}
\usepackage{lineno}
\usepackage{ulem}

% Theorem
\newtheorem{thm}{Theorem}[section]
\newtheorem{lem}[thm]{Lemma}
\newtheorem{prop}[thm]{Proposition}
\newtheorem{cor}[thm]{Corollary}
\newtheorem{defn}[thm]{Definition}
\newtheorem{rem}[thm]{Remark}

% \newcommand{\dt}{\partial_t}
\newcommand{\Lap}{\Delta}

\title{Test Paper}
%%%%%%%%%%%%%
\author{Keiichiro Kagawa\thanks{Bakada University, Corresponding author: gerogero7429@gmail.com}}
% \address{Bakada University}
%%%%%%%%%%%%%%%%

\begin{document}

\maketitle

\begin{abstract}
  In this paper, we are concerned with the existence of solutions for the Bakada equation.
\end{abstract}

\noindent
  {\bf AMS subject classification.}
  00X00.

%%%%%%%%%%%%%%% INTRODUCTION %%%%%%%%%%%%%%%%
\section{Introduction}\indent
  %%%%%%%%%%%%%%% Problem %%%%%%%%%%%%%%%
  In this paper, we discuss the initial boundary value problem 
  of the following Bakada equation with the homogeneous Dirichlet boundary condition.
  \begin{equation}\label{BE}
    \beta a = ka \Lap a,\ x\in\Omega,
  \end{equation}
  where $\Omega$ is a bounded domain in $\mathbb{R}^3$ with smooth boundary $\partial\Omega$;
  $a:[0, T]\times\Omega\to\mathbb{R}$ is an unknown function;
  $\beta$ and $k$ are positive constants.
    The Bakada equation was proposed by Kumasan \cite{Kumasan} in 1901 
    as a mathematical model to describe the ecology of bears in Hilbert space.

\section*{Acknowledgments}
  The author is supported by the Big Bear Fellowship {\#}BB01234.

%%%%%%%%%%%%%%% BIBLIOGRAPHY %%%%%%%%%%%%%%%%%%
\bibliography{ref}
\bibliographystyle{elsarticle-num}


\end{document}

